\section{Bureaux}

\subsection{getBureaux}
\shadowbox{
\verb|/parapheur/api/getBureaux|
}\\

Retourne la liste des bureaux dont l'utilisateur passé en paramètre est propriétaire ainsi que le nombre de dossier(s) dans présent(s) dans chaque corbeille (voir extrait~\ref{snip:getBureaux_out}) 

\begin{codesnippet}
\inputminted[frame=single,linenos,fontsize=\footnotesize]{javascript}{extraits/getBureaux_in.js}
\caption{getBureaux requête entrante}
\label{snip:getBureaux_in}
\end{codesnippet}

\subsection{getDossiersHeaders}
\shadowbox{
\verb|/parapheur/api/getDossiersHeaders|
}\\

\subsubsection{Entêtes de dossiers paginées}

Retourne une liste d'entêtes de dossiers issus de la bannette ``a-traiter''. (voir extraits~\ref{snip:getDossiersHeaders_in} et~\ref{snip:getDossiersHeaders_out}).

Lorsque l'on limite le nombre ($n$) de résultats \verb|getDossiersHeaders| retourne au plus $n + 1$ résultats afin d'indiquer une page suivante potentielle. Si \verb|pageSize| vaut 0 alors tout les résultats sont retournés.

\begin{codesnippet}
\inputminted[frame=single,linenos,fontsize=\footnotesize]{javascript}{extraits/getDossiersHeaders_in.js}
\caption{getDossiersHeaders requête entrante}
\label{snip:getDossiersHeaders_in}
\end{codesnippet}

\subsubsection{Filtres}
\label{Filtres}
Dans la requête des filtres il est possible d'utiliser les variables suivantes~\ref{table:filterable_fields} le type de la donnée est donné à titre indicatif car représenté sous forme de \verb|STRING| dans l'élément filtres de la requête.


\begin{figure}[H]
	\begin{center}
\begin{tabular}{c|c|c}
	\hline
	Nom & Type & Examples de valeurs \\
	\hline
	cm:name & STRING & re7 07\\
	ph:recuperable & BOOLEAN & false \\
	ph:reading-mandatory & BOOLEAN & false\\
	ph:signature-papier & BOOLEAN & false\\
	ph:tdtProtocole & STRING  & ACTES\\
	ph:dateLimite & DATE & null\\
	ph:soustypeMetier & STRING & Arrete du personel \\
	ph:confidentiel & BOOLEAN & false \\
	ph:termine & BOOLEAN & false \\
	ph:tdtNom & STRING & $S^2LOW$ \\
	ph:typeSignature & STRING & CMS\\
	ph:digital-signature-mandatory & BOOLEAN & false \\
	ph:sigFormat & STRING & PKCS$\sharp$7/single \\
	cm:modified & DATE & 2012-21-05 \\
	ph:typeMetier & STRING & Actes \\
	cm:created & DATE & 2012-17-01
	
\end{tabular}
\end{center}
\caption{Filtres potentiels}
\label{table:filterable_fields}

\end{figure}

Par exemple si l'on veut tout les dossiers commençant par ``Test''\footnote{On notera l'utilisation du caractère joker *} et ayant pour type ``HELIOS'' et sousType ``depense''
\begin{codesnippet}
\begin{minted}[frame=single,linenos,fontsize=\footnotesize]{javascript}
	"filtres": {
		"cm:name" : "Test*",
		"ph:typeMetier" : "HELIOS",
		"ph:soustypeMetier" : "depense"
	}
\end{minted}
\caption{Exemple de filtre à prédicats multiples}
\end{codesnippet}


\subsubsection{Filtres sur les champs de type date}

Les champs de type date peuvent être filtrés à l'aide d'un intervalle de date\footnote{Il est fort probable que cela fonctionne avec d'autres types énumérés par exemple les entiers}.

\begin{codesnippet}
\begin{minted}[frame=single,linenos,fontsize=\footnotesize]{javascript}
	"filtres": {
		"cm:created" : "[2012-06-01 TO NOW]",
	}
	
	"filtres": {
		"cm:created" : "[NOW TO 2012-06-01]",
	}
	
	"filtres": {
		"cm:created" : "[2012-06-01 TO 2012-06-06]",
	}
\end{minted}
\caption{Exemple de filtre sur des champs de type date}
\end{codesnippet}

\subsubsection{Filtres Complexes}
Il est possible de faire des filtres complexes contenant des clauses ``OR'' et ``AND''.

\begin{codesnippet}
\inputminted[frame=single,linenos,fontsize=\footnotesize]{javascript}{extraits/getDossiersHeaders_search_complex_in.js}
\caption{getDossiersHeaders Complex filters}
\label{snip:getTypologie_in}
\end{codesnippet}

\subsubsection{Spécification du Parent virtuel}

Par défaut les filtres retournent des résultats sur les dossiers provenant de la corbeille ``a-traiter''. Il est désormais possible de
choisir la corbeille (y compris virtuelle). Le tableau suivant~\ref{table:parent_values} récapitule les différentes corbeilles possibles.

\begin{figure}
\begin{center}
\begin{tabular}{c|p{7cm}}
	\hline
	Corbeille & Déscription \\
	\hline
        \verb|en-preparation| & contient les dossiers qui n'ont pas été encore émis  \\
        \verb|a-traiter| & contient les dossier que l'utilisateur doit traiter (signer, viser, etc ...) \\
        \verb|a-archiver| & contient les dossiers qui sont en fin de circuit ou à une étape d'archivage \\
        \verb|retournes| & contient les dossiers rejetés \\
        \verb|en-cours| & contient les dossiers emis par le proprietaire du bureau et qui suivent un circuit \\
        \verb|a-venir| & contient les dossiers qui vont arriver dans le bureau \\
        \verb|recuperables| & contient les dossiers récupérables (ie que l'on vient de viser) \\
        \verb|secretariat| & contient les dossiers envoyés au secretariat \\
        \verb|en-retard| & contient les dossiers dont la date limite < date du jour (valable uniquement pour le bureau emetteur) \\
        \verb|a-imprimer| & contient les dossiers à imprimer par la secretaire


\end{tabular}
\end{center}
\caption{Description des valeurs possibles de PARENT}
\label{table:parent_values}
\end{figure}


\subsection{getCircuit}
\shadowbox{\verb|/parapheur/api/getCircuit|
}\\


Retourne le circuit associé au dossierRef passé en paramètre (voir extrait ~\ref{snip:getCircuit_out}) 

\begin{codesnippet}
\inputminted[frame=single,linenos,fontsize=\footnotesize]{javascript}{extraits/getCircuit_in.js}
\caption{getCircuit in}
\label{snip:getCircuit_in}
\end{codesnippet}


\subsection{getTypologie}
\shadowbox{\verb|/parapheur/api/getTypologie|
}\\

Retourne la typologie accessible au bureau passé en paramètre (voir extrait~\ref{snip:getTypologie_out}).

\begin{codesnippet}
\inputminted[frame=single,linenos,fontsize=\footnotesize]{javascript}{extraits/getTypologie_in.js}
\caption{getTypologie in}
\label{snip:getTypologie_in}
\end{codesnippet}
